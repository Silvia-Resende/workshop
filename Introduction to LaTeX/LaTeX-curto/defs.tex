% !TeX root = latex-curto.tex
% !TeX encoding = utf8

% Personaizando o beamer
\usetheme{default}
\usecolortheme{orchid}

\usepackage[utf8]{inputenc}
\usepackage{graphicx,color}
\usepackage{amsmath,amssymb,dsfont}
\usepackage{wasysym}
\usepackage{multicol,bm}
\usepackage{etex}
\usepackage{tikz}

\setbeamersize{description width=0cm}

\frenchspacing

% logotipo do TikZ -> \Tikz
\newcommand{\Tikz}{Ti\textit{k}Z}

% emulando o pacote dingbat
%\usepackage{dingbat}
\makeatletter
\providecommand{\arkfamily}{\fontencoding{U}\fontfamily{ark}\selectfont}
\providecommand{\ark@sym}[1]{{\arkfamily\symbol{#1}}}
\newcommand{\leftthumbsup}{\ark@sym{'125}}
\newcommand{\leftthumbsdown}{\ark@sym{'104}}
\makeatother

\newcommand{\imagem}[2][]{\includegraphics[#1]{./imagens/#2}}

\newcommand{\vantagem}{{\darkgreen{\leftthumbsup}}}
\newcommand{\desvantagem}{{\red{\leftthumbsdown}}}

\newcommand{\sen}{\operatorname{sen}}

\usepackage[brazil]{babel}

\usepackage{enumerate}

%\renewcommand{\definitionname}{Definição}


\newcommand{\red}[1]{\textcolor{red}{#1}}
\newcommand{\redunder}[1]{\textcolor{red}{\underline{#1}}}
\newcommand{\green}[1]{\textcolor[rgb]{0,0.4,0}{#1}}
\newcommand{\darkgreen}[1]{\textcolor[rgb]{0,0.35,0}{#1}}
\newcommand{\blue}[1]{\textcolor{blue}{#1}}
\newcommand{\purple}[1]{\textcolor{purple}{#1}}
\newcommand{\cyan}[1]{\textcolor{cyan}{#1}}
\newcommand{\brown}[1]{\textcolor{brown}{#1}}
\newcommand{\gray}[1]{\textcolor{gray}{#1}}
\newcommand{\black}[1]{\textcolor{black}{#1}}

\theoremstyle{example}
\newtheorem*{exemplo}{Exemplo}
\newtheorem*{exemplos}{Exemplos}
\newtheorem*{dica}{Dica}


%% macros para escrever em texto puro
\newcommand{\ttbackslash}{\char92}
\let\bs\ttbackslash
\newcommand{\ttabrechaves}{\char123}
\newcommand{\ttfechachaves}{\char125}
\let\ac\ttabrechaves
\let\fc\ttfechachaves
\newcommand{\ttporcento}{\char37}
\let\pc\ttporcento
\newcommand{\ttdolar}{\char36}
\let\dolar\ttdolar
\newcommand{\ttet}{\char38}
\let\et\ttet
\newcommand{\ttcerquinha}{\char35}
\let\num\ttcerquinha
\newcommand{\ttmaior}{\char62}
\let\maior\ttmaior
\newcommand{\ttmenor}{\char60}
\let\menor\ttmenor
\newcommand{\ttaspasimples}{\char39}
\let\aspa\ttaspasimples
\newcommand{\ttgrave}{\char96}
\let\grave\ttgrave
\newcommand{\ttus}{\char95}
\let\us\ttus


\newenvironment{code}{\medskip
  \noindent\mbox{}\hspace{1cm}\begin{minipage}[t]{10cm}\ttfamily\obeylines}
  {\end{minipage}}

\newenvironment{code*}{\begin{minipage}[t]{10cm}\ttfamily\obeylines}{\end{minipage}}

\makeatletter
\renewcommand{\definition}{\@thm {\let \thm@swap \@gobble \th@definition }{theorem}{Definição}}
\makeatother

\setbeamercovered{transparent=50}

\newcommand{\vecs}[1]{(#1_1,\dots,#1_n)}
\newcommand{\vecx}[2][n]{(#2_1,\dots,#2_{#1})}

\endinput


(progn

(fset 'frame
   [?\\ ?b ?e ?g ?i ?n ?\{ ?f ?r ?a ?m ?e ?\} return ?\\ ?e ?n ?d ?\{ ?f ?r ?a ?m ?e ?\} home return up tab ?\\ ?f ?r ?a ?m ?e ?t ?i ?t ?l ?e ?\{ ?\} left])

(defun codify (prefix)
  "Substitui código puro em sua versão tipografada com as macros do
  curso de LaTeX. Com argumento prefixo, rodeia com \texttt{...}."
  (interactive "P")
  (let* ((open-curly-chackets-command "\\\\ac")
	 (close-curly-chackets-command "\\\\fc")
	 (dollar-command "\\\\dolar")
         (circ-command "\\\\textasciicircum")
	 (percent-command "\\\\pc")
         (underscore-command "\\\\us")
         (et-command "\\\\et")
         (backslash-command "\\\\bs")
	 (replace-list (list
			 (cons "\\\\" "\\\\string\\\\")
			 (cons "{" (concat open-curly-chackets-command "{}"))
			 (cons "}" (concat close-curly-chackets-command "{}"))
			 ;; fixing brackets bug
			 (cons (concat open-curly-chackets-command "{"
				       close-curly-chackets-command "{}")
			       (concat open-curly-chackets-command
                               "{}"))
                         ;; fixing bug for \\
                         (cons "\\\\string\\\\\\\\string"
                               (concat backslash-command
                                       backslash-command "{}"))
			 (cons "\\$" (concat dollar-command "{}"))
			 (cons "&" (concat et-command "{}"))
			 (cons "\\^" (concat circ-command "{}"))
                         (cons "_" (concat underscore-command "{}"))
			 (cons "%\\([^\n]*\\)\n"
			       (concat "\\\\textit{"
				       percent-command
				       "{}\\1}\n"))
			 (cons " " "\\\\ ")
			 (cons "^\\\\ \\\\ \\\\ \\\\ " "    ")
			 (cons "\n" "\\\\\\\\\n")))
	 (is-region (and transient-mark-mode
			 mark-active))
	 (beg (set-marker (make-marker) (if is-region
					    (region-beginning) (point-min))))
	 (end (set-marker (make-marker) (if is-region
					    (region-end) (point-max))))
	 (region-str-com (buffer-substring beg end))
	 (region-str-changed (copy-sequence region-str-com))
	 (deactivate-mark nil)
	 (case-repace nil)
	 (case-fold-search nil))
					; commenting
					; reclacing in string
    (dolist (rep replace-list)
      (let* ((from (car rep))
	     (to (cdr rep)))
	(setq region-str-changed
	      (replace-regexp-in-string from to region-str-changed))))
    (save-excursion
      (delete-region beg end)
      (goto-char beg)
      (cond ((equal prefix '(4)) 	; insert subs in \texttt{ }
	     (insert "\\texttt{" region-str-changed "}"))
	    ((equal prefix '(16)) ; insert commented raw + subs in \texttt{ }
	     (setq region-str-com
		   (replace-regexp-in-string "^" "%" region-str-com))
	     (insert region-str-com "\n")
	     (insert "\\texttt{" region-str-changed "}"))
	    (t		; insert only subs
	     (insert region-str-changed))))
    (set-marker beg nil)
    (set-marker end nil)))

(local-set-key (kbd "C-c C-d") 'codify)

)  ; fim do progn

%%% Local Variables: 
%%% mode: latex
%%% TeX-master: "latex-curto"
%%% End: 
