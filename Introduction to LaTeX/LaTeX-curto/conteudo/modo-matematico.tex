% !TeX root = ../latex-curto.tex
% !TeX encoding = utf8

\part{Modo Matemático}


\begin{frame}
  \frametitle{Estilos principais do modo matemático}
 
  \begin{block}{Estilo em linha}% ( = \texttt{\string\textstyle})
    A fórmula fica misturada ao texto na mesma linha.
  \end{block}
    \begin{exemplo}[]
      Seja $f(x) =\int_0^x \frac{\sen x}{x} dx$ a área \dots
    \end{exemplo}
    
    \medskip


  \begin{block}{Estilo em destaque}% ( = \texttt{\string\displaystyle})
    A fórmula se separa do texto, centralizada e com mais espaço.
  \end{block}
  \begin{exemplo}
    Seja
    \[
    f(x) =\int_0^x \frac{\sen x}{x} dx
    \] a área \dots
  \end{exemplo}
  


\end{frame}


\begin{frame}
  \frametitle{Modo matemático}

  \begin{block}{Estilo em linha}

    \begin{itemize}
    \item \texttt{\blue\$ ...\ \blue\$}
    \item \texttt{\blue{\string\(} ...\ \blue{\string\)}}
    \end{itemize}

  \end{block}

    \begin{exemplo}
        \texttt{A fórmula de Euler, dada por \blue{\dolar
          e\^{}\ac{}i\string\pi\fc{} + 1 = 0\dolar},\\ é considerada uma das
          mais bonitas fórmulas matemáticas.}

        \bigskip\hrule\bigskip

        A fórmula de Euler, dada por \blue{$e^{i\pi}+1=0$}, é considerada uma das
          mais bonitas fórmulas matemáticas.
    \end{exemplo}
\end{frame}

\begin{frame}
  \frametitle{Modo matemático}

  \begin{block}{Estilo destaque SEM numeração}
    \begin{itemize}
    \item \texttt{\blue{\string\[} ...\ \blue{\string\]}}
    \item \texttt{\blue{\string\begin\ac{}equation*\fc{}}\ ...\ \blue{\string\end\ac{}equation*\fc{}}}
    \end{itemize}
  \end{block}

\begin{exemplo}
  \texttt{A fórmula de Euler é dada por\\
  \blue{\string\[\\
  \mbox{}\ \ e\^{}\ac{}i\string\pi\fc{} + 1 = 0.\\
  \string\]}}

  \bigskip\hrule\bigskip

  A fórmula de Euler é dada por
  \[ \blue{e^{i\pi} + 1 = 0.}\]\smallskip
\end{exemplo}
\end{frame}

\begin{frame}
  \frametitle{Modo matemático}
  \begin{block}{Modo destaque COM numeração}
    \begin{itemize}
    \item \texttt{%\begin{equation} ... \end{equation}
        \blue{\string\begin\ac{}equation\fc{}}\ ...\ \blue{\string\end\ac{}equation\fc{}}}
%    \item equações de várias linhas
    \end{itemize}
  \end{block}

\begin{exemplo}
  \texttt{A fórmula de Euler é dada por\\
    \blue{\string\begin\ac{}equation\fc{}} \purple{\string\label\ac{}\underline{eq:\ euler}\fc{}}\\
      \mbox{}\ \  \blue{e\^{}\ac{}i\string\pi\fc{} + 1 = 0.}\\
      \blue{\string\end\ac{}equation\fc{}}\\
    ... Ver \purple{\string\eqref\ac{}\underline{eq:\ euler}\fc}.}

  \medskip\hrule\medskip

  A fórmula de Euler é dada por\blue{%
  \begin{equation} \label{eq: euler}
    e^{i\pi} + 1 = 0.
  \end{equation}}
  ... Ver \eqref{eq: euler}.
\end{exemplo}
\end{frame}


\section{Símbolos}

\begin{frame}
  \frametitle{Elementos simples}

  \begin{block}{Elementos simples}
    \begin{tabular}{lll}
      \green{Tipo} & \green{\TeX\ (modo matem.)} & \green{PDF} \\ %\hline
      Letras latinas & \blue{\tt a b x y z A B X Y} & $a b x y z A B X Y$\\
      %\hline
      Letras gregas minúsc. & \blue{\tt\string\alpha\ \string\delta} & $\alpha
      \delta$ \\ %\hline
      Letras gregas maiúsc. & \blue{\tt\string\Omega\ \string\Delta} & $\Omega
      \Delta$ \\ %\hline
      Outros símbolos & \blue{\tt\string\infty\ \string\exists}
      &$\infty\exists$\\%\hline
      & \blue{\texttt{\string\varnothing}} & $\varnothing$
    \end{tabular}
  \end{block}


\bigskip

  Mais:

  \begin{itemize}
  \item Apostila \LaTeX{} de A a B, p.\ 39.
  \item Compreensive \LaTeX\ symbols list (CTAN) \texttt{symbols-a4.pdf}
  \end{itemize}

\end{frame}

\begin{frame}
  \frametitle{Ops...}

  \begin{alertblock}{Modo matemático não é itálico!}

    \centering

    \imagem[clip,scale=0.5,bb=140 665 460 720]{diferente1.pdf}

    \medskip\hrule\medskip

    \imagem[clip,scale=0.5,bb=140 665 460 720]{diferente2.pdf}

  \end{alertblock}

\end{frame}

\begin{frame}
  \frametitle{Relações binárias}
  \begin{block}{Relações binárias}
    \centering
    \begin{tabular}{lcllcllc}
      \blue{\texttt{\string=}} & $=$ &&
      \blue{\texttt{\string\neq}} & $\neq$ &&
      \blue{\texttt{\string\approx}} & $\approx$ \\
      \blue{\texttt{\string<}} & $<$ &&
      \blue{\texttt{\string>}} & $>$ &&
      \blue{\texttt{\string\in}} & $\in$ \\
      \blue{\texttt{\string\leq}} & $\leq$ &&
      \blue{\texttt{\string\geq}} & $\geq$ &&
      \blue{\texttt{\string\not\string\in}} & $\not\in$ \\
      \blue{\texttt{\string\subset}} & $\subset$ &&
      \blue{\texttt{\string\supset}} & $\supset$ &&
      \blue{\texttt{\string\perp}} & $\perp$
    \end{tabular}
  \end{block}

  \begin{block}{Operadores binários}
    \centering
    \begin{tabular}{llllllll}
      \blue{\texttt{\string\pm}} & $\pm$ &&
      \blue{\texttt{\string\times}} & $\times$ &&
      \blue{\texttt{\string\div}} & $\div$ \\
      \blue{\texttt{\string\cap}} & $\cap$ &&
      \blue{\texttt{\string\cup}} & $\cup$ &&
      \blue{\texttt{\string\cdot}} & $\cdot$ 
    \end{tabular}
  \end{block}

\bigskip

  Mais:

  \begin{itemize}
  \item Apostila \LaTeX{} de A a B, p.\ 38.
  \item Compreensive \LaTeX\ symbols list (CTAN) \texttt{symbols-a4.pdf}
  \end{itemize}
\end{frame}


\begin{frame}
   \frametitle{Delimitadores}
  \begin{block}{Delimitadores}
    \centering
    \begin{tabular}{lcllc}
      \blue{\texttt{( )}} & $\bigl(\,\bigr)$ &&
      \blue{\texttt{[ ]}} & $\bigl[\, \bigr]$ \\[2mm]
      \blue{\texttt{\string| \string|}} & $\bigl|\,\bigr|$ &&
      \blue{\texttt{\string\| \string\|}} & $\bigl\|\, \bigr\|$ \\[2mm]
      \blue{\texttt{\string\langle\ \string\rangle}} & $\bigl\langle\,\bigr\rangle$ &&
      \blue{\texttt{\string\lbrace\ \string\rbrace}} & $\bigl\lbrace\,\bigr\rbrace$
    \end{tabular}
  \end{block}

  \def\x{\dfrac12}

  \begin{block}{Tamanhos \hfill (obs: \texttt{\string\x} = \texttt{\string\dfrac12})}
    \centering
    \begin{tabular}{lcllc}
      \blue{\texttt{(\ \black{\string\x}\ )}} & $( \x )$ &&
      \blue{\texttt{\string\left(\ \black{\string\x}\ \string\right)}} & $\left( \x \right)$  \\
      \blue{\texttt{\string\bigl(\ \black{\string\x}\ \string\bigr)}} & $\bigl( \x \bigr)$ &&
      \blue{\texttt{\string\Bigl(\ \black{\string\x}\ \string\Bigr)}} & $\Bigl( \x \Bigr)$ \\
      \blue{\texttt{\string\biggl(\ \black{\string\x}\ \string\biggr)}} & $\biggl( \x \biggr)$ &&
      \blue{\texttt{\string\Biggl(\ \black{\string\x}\ \string\Biggr)}} & $\Biggl( \x \Biggr)$
    \end{tabular}
  \end{block}
\end{frame}

\begin{frame}
  \frametitle{Fontes matemáticas}
  \begin{block}{Caligráficas}
    \texttt{\blue{\string\mathcal}\ac{}\textit{letra}\fc}
    \[
    \mathcal{A}\,\mathcal{B}\,\mathcal{C}\,\mathcal{D}\,\mathcal{E}\,\mathcal{F}\,\mathcal{G}\,\mathcal{H}\,\mathcal{I}\,\mathcal{J}\,\mathcal{K}\,\mathcal{L}\,\mathcal{M}\,\mathcal{N}\,\mathcal{O}\,\mathcal{P}\,\mathcal{Q}\,\mathcal{R}\,\mathcal{S}\,\mathcal{T}\,\mathcal{U}\,\mathcal{V}\,\mathcal{W}\,\mathcal{X}\,\mathcal{Y}\,\mathcal{Z}
    \]
  \end{block}

  \begin{block}{Blackboard Bold \hfill(\texttt{\string\usepackage\ac{}amssymb\fc{}})}
    \texttt{\blue{\string\mathbb}\ac{}\textit{letra}\fc}
    \[
    \mathbb{A}\,\mathbb{B}\,\mathbb{C}\,\mathbb{D}\,\mathbb{E}\,\mathbb{F}\,\mathbb{G}\,\mathbb{H}\,\mathbb{I}\,\mathbb{J}\,\mathbb{K}\,\mathbb{L}\,\mathbb{M}\,\mathbb{N}\,\mathbb{O}\,\mathbb{P}\,\mathbb{Q}\,\mathbb{R}\,\mathbb{S}\,\mathbb{T}\,\mathbb{U}\,\mathbb{V}\,\mathbb{W}\,\mathbb{X}\,\mathbb{Y}\,\mathbb{Z}
    \]
  \end{block}

  \begin{block}{Double Stroke \hfill(\texttt{\string\usepackage\ac{}dsfont\fc{}})}
    \texttt{\blue{\string\mathds}\ac{}\textit{letra}\fc}
    \[
    \mathds{A}\,\mathds{B}\,\mathds{C}\,\mathds{D}\,\mathds{E}\,\mathds{F}\,\mathds{G}\,\mathds{H}\,\mathds{I}\,\mathds{J}\,\mathds{K}\,\mathds{L}\,\mathds{M}\,\mathds{N}\,\mathds{O}\,\mathds{P}\,\mathds{Q}\,\mathds{R}\,\mathds{S}\,\mathds{T}\,\mathds{U}\,\mathds{V}\,\mathds{W}\,\mathds{X}\,\mathds{Y}\,\mathds{Z}
    \]
  \end{block}

\end{frame}

\section{Construções}

\begin{frame}
  \frametitle{Índices e expoentes}

  \begin{block}{Índices e expoentes}
    \centering
    \begin{tabular}{llllll}
      \blue{\texttt{x\^{}2}} & $x^2$ &&&
      \blue{\texttt{x\us n}} & $x_n$ \\
      \blue{\texttt{x\^{}2\us n}} & $x_n^2$ &&&
      \blue{\texttt{x\us\ac{}n\us k\fc}} & $x_{n_k}$ \\
      \blue{\texttt{x\us n\us k}} & \alert{erro}
    \end{tabular}
  \end{block}

  \begin{block}{Somatórios e integrais}
    \texttt{\blue{\string\sum}\green{\us\ac{}i=1\fc}\purple{\^{}\string\infty{}}
      \string\frac\ac{}1\fc{}\ac{}n\^{}2\fc{}\ =
      \string\frac\ac{}\string\pi\^{}2\fc{}\ac{}6\fc{}}
%    \vspace{-3mm}
    \[\sum_{i=1}^\infty \frac{1}{n^2} = \frac{\pi^2}{6}\]
%\vspace{-2mm}

\hrule\medskip

    \texttt{\blue{\string\int}\green{\us{}0}\purple{\^{}\string\pi}\ \string\sen\ x\string\,dx
      = 2}
    %\vspace{-3mm}
    \[\int_0^\pi \sen x\,dx=2\]
  \end{block}

\end{frame}

\begin{frame}
  \frametitle{Frações}

  \begin{block}{\texttt{\string\frac\ac{a}\fc\ac{b}\fc}}
    \begin{columns}
      \column{1cm}
      \texttt{%\frac{a}{b}
        \blue{\string\frac}\ac{}\textit{a}\fc{}\ac{}\textit{b}\fc{}}
      \column{5cm}
      \begin{tabular}{lcllc}
        Estilo em linha &&  $\frac ab$ \\[2mm]
        Estilo destaque &&  $\dfrac ab$
      \end{tabular}
    \end{columns}

  \end{block}


  \begin{block}{Forçando modo}

    \begin{itemize}
    \item \blue{\texttt{\string\tfrac}} $\to$ fração estilo em linha
      \quad\gray{\small(t $\to$ \texttt{\bs \underline{t}extstyle})}
    \item \blue{\texttt{\string\dfrac}} $\to$ fração estilo destaque
      \quad\gray{\small(d $\to$ \texttt{\bs \underline{d}isplaystyle})}
    \end{itemize}

  \end{block}

  \begin{exemplo}\centering
    \texttt{\string\[\ \string\int\ \blue{\string\frac\ac{}1\fc{}\ac{}x\fc{}}\ dx\ =\string\int\ \green{\string\tfrac\ac{}1\fc{}\ac{}x\fc{}}\ dx\ \string\]}
    \medskip\hrule
    \[ \int \blue{\frac{1}{x}} dx = \int \green{\tfrac{1}{x}} dx \]
  \end{exemplo}

\end{frame}

\begin{frame}
  \frametitle{Raízes}

  \begin{block}{Raízes}
    \centering
    \begin{tabular}{lcllc}
      \texttt{\blue{\string\sqrt}\ac{}x\fc{}} &&  $\sqrt x$ \\[2mm]
      \texttt{\blue{\string\sqrt}\red{[3]}\ac{}x\fc{}} &&  $\sqrt[3]{x}$
    \end{tabular}

  \end{block}

    \begin{exemplo}
      \texttt{%\sqrt{3-2\sqrt2} = \sqrt2-1
        \string\sqrt\ac{}3-2\string\sqrt2\fc{}\ =\ \string\sqrt2-1}
      \medskip\hrule\medskip
      \[\sqrt{3-2\sqrt2} = \sqrt2-1 \]
    \end{exemplo}

\end{frame}

\begin{frame}
  \setbeamercovered{transparent=0}

  \frametitle{Funções, limites, \dots}

  \begin{block}{Funções, limites, \dots}
    \centering
    \begin{tabular}{llllllll}
      \blue{\texttt{\string\cos}} & $\cos$ &&
      \blue{\texttt{\string\sin}} & $\sin$ &&
      \blue{\texttt{\string\tan}} & $\tan$ \\
      \blue{\texttt{\string\det}} & $\det$ &&
      \blue{\texttt{\string\log}} & $\log$ &&
      \blue{\texttt{\string\exp}} & $\exp$
    \end{tabular}
  \end{block}

  \begin{alertblock}{\texttt{\string\sen} não existe!}
    \texttt{\purple{\string\newcommand}\ac{}\blue{\string\sen}\fc{}\ac{}\green{\string\operatorname}\ac{}sen\fc{}\fc{}}
  \end{alertblock}

  \begin{exemplo}
    %\lim_{x\to 0} \frac{\sen x}{x} = 1
    \texttt{\string\lim\us\ac{}x\string\to\ 0\fc{}\
      \string\frac\ac{}\blue{\string\sen}\ x\fc{}\ac{}x\fc{}\ =\ 1}

    \medskip\hrule\medskip
    \[\lim_{x\to 0} \frac{\sen x}{x} = 1\]
  \end{exemplo}

\end{frame}

\begin{frame}
  \frametitle{Matrizes}

  \begin{exemplo}
    \begin{columns}[c]
      \column{4cm}{\ttfamily
%      \begin{pmatrix}
%         1 & 2 & 3 \\
%        -1 & 0 & 5 \\
%         0 & 3 & 4
%      \end{pmatrix}
        \blue{\string\begin\ac{}pmatrix\fc{}}\\
          \ \ \ 1\ \red{\et}\ 2\ \red{\et}\ 3\ \purple{\bs\bs}\\
          \ \ -1\ \red{\et}\ 0\ \red{\et}\ 5\ \purple{\bs\bs}\\
          \ \ \ 0\ \red{\et}\ 3\ \red{\et}\ 4\\
        \blue{\string\end\ac{}pmatrix\fc{}}}
      \column{3cm}
      \[
      \begin{pmatrix}
         1 & 2 & 3 \\
        -1 & 0 & 5 \\
         0 & 3 & 4
      \end{pmatrix}
      \]
    \end{columns}
  \end{exemplo}


  \begin{exemplo}
    {\small
    \texttt{Seja\ \dolar{}A=\brown{\string\left(}\blue{\string\begin\ac{}smallmatrix\fc{}}\\
    \ \ \ \ \ \ \ \ \ \ \ \ \ \ \ \ \ 0\ \red{\et}\ 1\ \purple{\bs\bs} -1\ \red{\et}\ 0\\
    \ \ \ \ \ \ \ \ \ \ \ \ \ \ \ \blue{\string\end\ac{}smallmatrix\fc{}}\brown{\string\right)}\dolar{}\ a\ matriz...}}

    \medskip\hrule\medskip

    Seja \blue{$A=\left(\begin{smallmatrix}
        0 & 1 \\ -1 & 0
      \end{smallmatrix}\right)$} a matriz...    
  \end{exemplo}

\end{frame}

\section{Fórmulas de várias linhas}

%\setcounter{equation}{0}

\begin{frame}
  \frametitle{Ambientes de várias linhas}


  \begin{block}{Alinhado}
      \texttt{%
%       \begin{align}
%         a_1 & = b_1 + c_1 \label{eq1}\\
%         a_2 & = b_2 + c_2
%                -d_2 + e_2 \nonumber
%       \end{align}
    \ \ \ \blue{\string\begin\ac{}align\fc{}}\\
    \ \ \ \ \ a\us{}1\ \red{\et}{}\ =\ b\us{}1\ +\ c\us{}1\
    \purple{\string\label\brown{\ac{}eq:\ align\fc{}}}\ \red{\bs\bs}{}\\
    \ \ \ \ \ a\us{}2\ \red{\et}{}\ =\ b\us{}2\ +\ c\us{}2\\
    \ \ \ \ \ \ \ \ \ \ \ \ -d\us{}2\ +\ e\us{}2\ \green{\string\nonumber}\\
    \ \ \ \blue{\string\end\ac{}align\fc{}}\\
    \ \ \ Segue da equação \purple{\string\eqref{\brown{\ac{}eq:\
          align\fc}}} ...}

\medskip\hrule

      \begin{align}
        a_1 &= b_1 + c_1 \label{eq: align} \\
        a_2 &= b_2 + c_2
              -d_2 + e_2 \nonumber
      \end{align}
      Segue da equação \eqref{eq: align} \dots
  \end{block}
\end{frame}

\begin{frame}
  \frametitle{Ambientes de várias linhas}
  \begin{block}{Centralizado}
      \texttt{%
%       \begin{gather}
%         a_1 & = b_1 + c_1 \label{eq1}\\
%         a_2 & = b_2 + c_2
%                -d_2 + e_2 \nonumber
%       \end{gather}
    \ \ \ \blue{\string\begin\ac{}gather\fc{}}\\
    \ \ \ \ \ a\us{}1\ =\ b\us{}1\ +\ c\us{}1\
    \purple{\string\label\brown{\ac{}eq:\ gather\fc{}}}\ \red{\bs\bs}{}\\
    \ \ \ \ \ a\us{}2\ =\ b\us{}2\ +\ c\us{}2\\
    \ \ \ \ \ \ \ \ \ \ -d\us{}2\ +\ e\us{}2\ \green{\string\nonumber}\\
    \ \ \ \blue{\string\end\ac{}gather\fc{}}\\
    \ \ \ Segue da equação \purple{\string\eqref{\brown{\ac{}eq:\
          gather\fc}}} ...}

\medskip\hrule

      \begin{gather}
        a_1 = b_1 + c_1 \label{eq: gather} \\
        a_2 = b_2 + c_2
              -d_2 + e_2 \nonumber
      \end{gather}
      Segue da equação \eqref{eq: gather} \dots
  \end{block}
\end{frame}

\begin{frame}
  \frametitle{Numeração e referência}

  \begin{block}{Numero ou não?}
    \centering
    \begin{tabular}{lll}
      \green{COM numeração} && \green{SEM numeração}\\ %\hline
      \blue{\texttt{equation}} && \blue{\texttt{equation\purple{*}}} \\
      \blue{\texttt{align}} && \blue{\texttt{align\purple{*}}} \\
      \blue{\texttt{gather}} && \blue{\texttt{gather\purple{*}}}
    \end{tabular}
  \end{block}

\end{frame}


%%% Local Variables: 
%%% mode: latex
%%% TeX-master: "../latex-curto"
%%% End: 
