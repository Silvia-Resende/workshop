% !TeX root = ../latex-curto.tex
% !TeX encoding = utf8

\section{Teoremas}

\begin{frame}
  \frametitle{Teoremas, definições, etc}

  \begin{block}{Ambientes para teoremas, definições, ...}

    \begin{itemize}
    \item preâmbulo:
      \texttt{\purple{\string\usepackage}\blue{\ac{}amsthm\fc}}\medskip

    \item Tipo:

      \begin{ttfamily}\small
        \green{\string\theoremstyle}\blue{\ac{}theorem\fc{}}\ \ \ \ \gray{\textit{\footnotesize\pc{}\ titulo\ negrito,\ corpo\ itálico}}\\
        \green{\string\theoremstyle}\blue{\ac{}definition\fc{}}\ \gray{\textit{\footnotesize\pc{}\ titulo\ negrito,\ corpo\ normal}}\\
        \green{\string\theoremstyle}\blue{\ac{}remark\fc{}}\ \ \ \ \ \gray{\textit{\footnotesize\pc{}\
          titulo\ itálico,\ corpo\ normal}}
      \end{ttfamily}

    \item Declarar ambientes tipo teorema:\smallskip

      \texttt{\small\purple{\string\newtheorem}\blue{\ac{}amb\fc{}}\green{\ac{}Nome\fc{}}\brown{[contador-superior]}}
      
      ou

      \texttt{\small\purple{\string\newtheorem}\blue{\ac{}amb\fc{}}\brown{[numerar-como-amb2]}\green{\ac{}Nome\fc{}}}

    \end{itemize}
    
  \end{block}

\end{frame}

\begin{frame}
  \frametitle{Teoremas, definições, etc}

  \begin{exemplo}[no cabeçalho]
    \bigskip

    \begin{ttfamily}\small
      \green{\string\theoremstyle}\blue{\ac{}theorem\fc{}}\\
      \purple{\string\newtheorem}\blue{\ac{}teo\fc{}}\green{\ac{}Teorema\fc{}}\brown{[chapter]}\\
      \purple{\string\newtheorem}\blue{\ac{}lema\fc{}}\brown{[teo]}\green{\ac{}Lema\fc{}}\\
      \ \\
      \green{\string\theoremstyle}\blue{\ac{}definition\fc{}}\\
      \purple{\string\newtheorem}\blue{\ac{}defi\fc{}}\brown{[teo]}\green{\ac{}Definição\fc{}}\\
    \end{ttfamily}

  \end{exemplo}

  {\small Uso no próximo slide...}
  
\end{frame}

\begin{frame}
  \frametitle{Teoremas, definições, etc}

  \begin{exemplo}[no corpo do documento]
    \begin{ttfamily}\footnotesize
%      \gray{\textit{\pc{}\ já\ no\ documento\ ...}}\\
      \purple{\string\chapter}\blue{\ac{}Teoria\ dos\ números\fc{}}\\
      \ \\
      \blue{\string\begin\ac{}defi\fc{}}\brown{[Terno\ pitagórico]}\\
        \ \ Um\ \string\emph\ac{}terno\ pitagórico\fc{}\ é\ formado\ por\ três\\
        \ \ números\ naturais\ \dolar{}a\dolar{},\ \dolar{}b\dolar{}\ e\ \dolar{}c\dolar{}\ tais\ que\ \dolar{}a\textasciicircum{}2+b\textasciicircum{}2=c\textasciicircum{}2\dolar{}.\\
        \blue{\string\end\ac{}defi\fc{}}\\
      \ \\
      \blue{\string\begin\ac{}teo\fc{}}\brown{[Fermat-Wiles]}\ \purple{\string\label}\brown{\ac{}teo:\ ultimo\ teo\ fermat\fc{}}\\
        \ \ \ Não\ existe\ nenhum\ conjunto\ de\ inteiros\ positivos\\
        \ \ \ \dolar{}x\dolar{},\ \dolar{}y\dolar{},\ \dolar{}z\dolar{}\ e\ \dolar{}n\dolar{},\ com\ \dolar{}n>2\dolar{},\ tais\ que\ \dolar{}x\textasciicircum{}n+y\textasciicircum{}n=z\textasciicircum{}n\dolar{}.\\
        \blue{\string\end\ac{}teo\fc{}}\\
      \ \\
      \blue{\string\begin\ac{}proof\fc{}}\\
        \ \ Seja\ \dolar{}\string\Delta\ ABC\dolar{}\ um\ triângulo\ retângulo...\\
        \blue{\string\end\ac{}proof\fc{}}
    \end{ttfamily}

  \end{exemplo}

  {\small Resultado no próximo slide...}
  
\end{frame}

\begin{frame}
    \frametitle{Teoremas, definições, etc}

    \begin{exemplo}\medskip
      \textbf{\Large Capítulo 1 \\[2mm] Teoria dos números} \bigskip\bigskip

      \small

      \textbf{Definição 1.1 (Terno pitagórico).}  
      Um \emph{terno pitagórico} é formado por três números naturais $a$,
      $b$ e $c$ tais que $a^2+b^2=c^2$.\bigskip
      
      \textbf{Teorema 1.2 (Fermat-Wiles).} 
      \textit{Não existe nenhum conjunto de inteiros positivos $x$, $y$, $z$ e
      $n$, com $n>2$, tais que}
      \[ x^n+y^n=z^n. \]
      
      \textit{Demonstração.} 
        Seja $\Delta ABC$ um triângulo retângulo... \hfill $\square$
      

    \end{exemplo}
  
\end{frame}


%%% Local Variables: 
%%% mode: latex
%%% TeX-master: "../latex-curto"
%%% End: 
