% !TeX root = ../latex-curto.tex
% !TeX encoding = utf8

\section{Bibliografia}

\begin{frame}
  \frametitle{Bibliografia}

  \begin{block}{Jeitos de implementar a bibliografia}
    \begin{itemize}
    \item \vantagem\ automático
    \item \desvantagem\ manual

    \end{itemize}

  \end{block}
\end{frame}

\begin{frame}
  \frametitle{Bibliografia manual}

  \begin{block}{Usando bibliografia manual \leftthumbsdown}
    \begin{itemize}
    \item Formata-se as entradas manualmente\\
      usando o ambiente \blue{\texttt{thebibliography}}\\
      em que cada entrada começa com
      \texttt{\green{\string\bibitem}\purple{\ac{}\textit{label}\fc{}}}\medskip
    \item \texttt{\blue{\string\cite}\purple{\ac{}label\fc{}}} no
      texto para citar
    \end{itemize}
  \end{block}

  \begin{alertblock}{Cuidado}
    Formatação manual é suscetível a inconsistências.
  \end{alertblock}
\end{frame}

\begin{frame}
  \frametitle{Bibliografia automática}

  \begin{block}{Usando Bib\TeX\ \leftthumbsup}
    \begin{itemize}
    \item Mantém-se um arquivo pessoal com extensão
      \blue{\texttt{.bib}}\\
      Ex: \purple{\texttt{teixeira.bib}}
    \item No arquivo \texttt{.bib}, cada entrada tem um \purple{\texttt{label}}.
    \item No final do documento, inclui-se as linhas
      \begin{quote}\upshape
        \ttfamily
        \blue{\string\bibliographystyle}\brown{\ac{}$\overbrace{\text{acm}}^{\text{estilo}}$\fc{}}\\[2mm]
        \blue{\string\bibliography}\redunder{\ac{}teixeira\fc{}}
      \end{quote}
    \item \texttt{\blue{\string\cite}\purple{\ac{}label\fc{}}} no
      texto para citar
    \end{itemize}
  \end{block}

\end{frame}

\begin{frame}
  \frametitle{Entradas no arquivo \texttt{.bib}}

  \begin{exemplo}
    A maioria das obras e artigos tem a entrada Bib\TeX{} pronta.

    \begin{itemize}
    \item No \blue{MathSciNet (\url{www.ams.org/mathscinet})},\\
      procurar obra
    \item Na página da obra, tem uma caixa de combo\\
      \ \ \ \ \ \ {\textit{Select alternative format}}
    \item Escolha \textbf{Bib\TeX}
    \item Mude o label à escolha
      e inclua no \texttt{.bib}
    \end{itemize}
  \end{exemplo}
\end{frame}

%%% Local Variables: 
%%% mode: latex
%%% TeX-master: "../latex-curto"
%%% End: 
