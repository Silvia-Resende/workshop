% !TeX root = ../latex-curto.tex
% !TeX encoding = utf8

\section{Inserindo imagens}

\begin{frame}
  \frametitle{Inserindo imagens}
  
  \begin{block}{}
    \ttfamily\blue{\string\usepackage}\ac\purple{graphic\underline{x}}\fc\quad
    \gray{\textit{\pc\ no cabeçalho}}\\
    \medskip\par
    \blue{\string\includegraphics}\brown{[ajustes]}\ac{}\purple{\textit{arquivo}}\fc{}

  \end{block}

  \begin{block}{Principais ajustes}
    \begin{itemize}
    \item \brown{\texttt{scale=\textit{número}}} redimensionar a imagem
    \item \brown{\texttt{width=\textit{tamanho}}} comprimento
    \item \brown{\texttt{height=\textit{tamanho}}} altura
    \end{itemize}
  \end{block}

\end{frame}


\begin{frame}
  \frametitle{Exemplo de inserção}

   \texttt{\blue{\string\includegraphics}\brown{[width=2cm]}\purple{\ac{}smiley.pdf\fc{}}}
\qquad
      \raisebox{-0.5\height}{\imagem[width=1.5cm]{smiley.pdf}}

\bigskip

  \begin{block}{Tipos de arquivos possíveis de incluir}
    \begin{itemize}
    \item pdf
    \item jpg
    \item png
    \end{itemize}
  \end{block}

\end{frame}

\section{Figuras}

% \begin{frame}
%   \frametitle{Exemplo de tabelas}

%   \begin{exemplo}

%     \begin{columns}[c]
%       \column{6cm}
%       \texttt{\blue{\string\begin\ac{}tabular\fc{}}\redunder{\ac{}|c|r|l|\fc{}}\\
%           \ \ \green{\string\hline}\\
%           \ \ a\ \ \ \red{\et}\ bb\ \ \red{\et}\ ccc\ \purple{\bs\bs}\ \green{\string\hline}\\
%           \ \ bb\ \ \red{\et}\ ccc\ \red{\et}\ a\ \ \ \purple{\bs\bs}\ \green{\string\hline}\\
%           \ \ ccc\ \red{\et}\ a\ \ \ \red{\et}\ bb\ \ \purple{\bs\bs}\ \green{\string\hline}\\
%           \blue{\string\end\ac{}tabular\fc{}}}
%       \column{3cm}
%       \begin{tabular}{|c|r|l|}
%         \hline
%         a   & bb  & ccc \\\hline
%         bb  & ccc & a   \\\hline
%         ccc & a   & bb   \\\hline
%       \end{tabular}
%     \end{columns}

%   \end{exemplo}
% \end{frame}

\begin{frame}
  \frametitle{Figuras e tabelas}

  \begin{block}{Elementos ``flutuantes''}
    \begin{itemize}
    \item figuras ou tabelas
    \item podem ser grandes\\
      $\to$ isto dificulta seu posicionamento na página
    \item $\therefore$ figuras e tabelas podem \green{deslocar-se na
        página}\\
      $\to$ são \purple{flutuantes}
    \end{itemize}
  \end{block}

\end{frame}

\begin{frame}
  \frametitle{Figuras}

  \begin{block}{Elementos das figuras (ambiente \texttt{figure})}
    \medskip
%    \begin{figure}[posição]
%
%      (conteúdo da figura)
%
%      \caption{legenda}
%      \label{fig: label}
%    \end{figure}
    \blue{\texttt{\string\begin\ac{}figure\fc{}\brown{[lista-de-posições]}\
          \ \gray{\small\textit{\% pos:\ h,t,b,p}}}\\[2mm]
    \ \ \ \ \ }\red{\textbf{(conteúdo\ da\ figura)}}\\[2mm]
    \texttt{\ \ \purple{\string\caption\ac{}\textit{Legenda}\fc{}}\\
     \ \ \ \ \gray{\textit{\% \string\label{} SEMPRE depois do
        \string\caption{} !!}}\\
    \ \ \purple{\string\label\ac{}\textit{fig:\ label}\fc{}}\\
    \blue{\string\end\ac{}figure\fc{}}}
    \medskip
  \end{block}

\begin{block}{Posições}
  \begin{description}
  \item[h] = here = aqui
  \item[t] = top = topo da página
  \item[b] = bottom = pé da página
  \item[p] = page = em página separada
  \end{description}
\end{block}
\end{frame}

\begin{frame}
  \frametitle{Exemplo de figura (inserindo imagem)}

  \begin{exemplo}
    \medskip
    \quad
    \begin{minipage}{10cm}
    \texttt{\purple{\string\usepackage}\blue{\ac{}graphicx\fc}
      \qquad
      \gray{\pc\ no preâmbulo}}
    \bigskip
    
    \begin{ttfamily}\small
        \green{\string\begin\ac{}figure\fc{}}\red{[hb]}\\
          \mbox{}\ \ \purple{\string\centering}\\
          \mbox{}\ \ \blue{\string\includegraphics}\brown{[width=2cm]}\purple{\ac{}smiley.pdf\fc{}}\\
          \mbox{}\ \ \purple{\string\caption}\brown{\ac{}Sorria,\ você\ NÃO\ está\ sendo\ filmado.\fc{}}\\
          \mbox{}\ \ \purple{\string\label}\brown{\ac{}fig:\ sorria\fc{}}\\
          \green{\string\end\ac{}figure\fc{}}
      \end{ttfamily}
    \end{minipage}
    
    \begin{figure}[hb]
      \centering
      \imagem[width=1.5cm]{smiley.pdf}
      \caption{Sorria, você NÃO está sendo filmado.}
      \label{fig:smiley}
    \end{figure}

  \end{exemplo}

\end{frame}

%%% Local Variables: 
%%% mode: latex
%%% TeX-master: "../latex-curto"
%%% End: 
